\section{گرامر های فایل 
\lr{bench}
}

پیش از آنکه به بررسی خواندن ورودی ها و تزریق آن ها به شبکه کنیم، نیاز است به بررسی گرامر های فایل های 
\lr{bench}
بپردازیم. برای نمونه فایل 
\lr{c17.bench}
به شکل زیر است:

\small{\lr{\lstinputlisting[numbers=left, breaklines=true]{../my_bench_files/c17.bench}}}

در این پروژه، برای این که چنین فایلی را پردازش 
\LTRfootnote{parse}
کنیم، از کتابخانه 
\lr{\textbf{pyparsing}}
استفاده می کنیم. گرامرهای فایل های 
\lr{bench}
را می توان به شکل زیر تعریف کرد: 
\small {\lr{\lstinputlisting[numbers=left, language=Python, breaklines=true, firstline=3]{../project/grammers/bench_grammer.py}}}

در واقع به ازای هر گرامر موجود، در خروجی یک لیست از اعداد برای ما تولید می کند و آن را توسط تابع دیگری که مسئولیت ساخت شبکه را دارد، تزریق می کنیم.


