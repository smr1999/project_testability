\section{شبیه ساز اشکال استنتاجی}
پس از اینکه گرامر های مربوط به فایل های 
\lr{bench}
خوانده شد و در ادامه گیت ها تشکیل و اتصال سیم ها با گیت ها برقرار شد و ورودی ها به سطح صفر مدار (یعنی پایانه های ورودی) تزریق می شود و در هر گام محاسبات روی گیت ها انجام شد و روی سیم های خروجی آن منتشر شد، نیاز است الگوریتم استنتاجی را اعمال کنیم. 

بر اساس صورت پروژه نیاز است که این الگوریتم را تنها برای منطق دو مقداری 0 و 1 اعمال کنیم و از منطق چهار مقداری که برای مرحله قبل بود صرف نظر کنیم. 

اگر بخواهیم این الگوریتم شبیه سازی اشکال استنتاجی را خلاصه کنیم، می توان به شکل زیر عمل کنیم:

\begin{enumerate}
	\item 
	\textbf{گیت BUFFER و NOT و FANOUT}
	: در واقع این گیت ها هر اشکال ورودی خود را به خروجی (ها)ی خود منتشر می کنند و در نهایت با فالت مربوط به سیم خروجی خود اجتماع می گیرند.
	
	\item
	\textbf{گیت AND و NAND}
	: برای این گیت ها به این صورت است که اگر مقدار خود گیت 1 (0 برای NAND) باشد، هر کدام از ورودی ها تغییر کند، خروجی تغییر می کند لذا لیست اشکال نهایی به صورت اجتماع لیست اشکال ورودی ها تعریف می شود. ولی اگر مقدار خود گیت 0 (1 برای NAND) باشد، برای تغییر خروجی ، نیاز است تمامی ورودی های صفر به یک تغییر کنند به شرطی که این تغییر روی سایر ورودی های یک تاثیری نداشته باشد. لذا لیست اشکال خروجی نهایی به صورت اشتراک لیست اشکال ورودی های صفر منهای اجتماع لیست اشکال ورودی های یک، تعریف می شود. البته باید توجه داشت که در نهایت لیست اشکال خود سیم خروجی را نیز به لیست اشکال های آن سیم اضافه کنیم.
	
	\item 
	\textbf{گیت OR و NOR}
	: برای این گیت ها در صورتی که مقدار خود گیت 0 (1 برای NOR) باشد، هر کدام از ورودی های گیت تغییر کنند، خروجی تغییر می کند. لذا لیست اشکال نهایی به صورت اجتماع اشکال های ورودی تعریف می شود. ولی اگر مقدار خود گیت 1 (0 برای NOR) باشد، برای تغییر خروجی، نیاز است تمام ورودی های یک به صفر تبدیل شوند به شرطی که این تغییر روی سایر ورودی های صفر تاثیری نداشته باشد. لذا لیست اشکال خروجی نهایی به صورت اشتراک لیست اشکال ورودی های یک منهای لیست اشکال ورودی های صفر تعریف می شود. البته همانند حالت قبل ، باید توجه داشت که در نهایت لیست اشکال خود سیم خروجی را نیز به لیست اشکال های آن سیم اضافه کنیم.
	
	\item
	\textbf{گیت XOR و XNOR}
	: لیست اشکال این نوع گیت ها به صورت دیگری برخلاف حالات قبل، تعریف می شود. لیست اشکال این گیت ها به این صورت که تعریف می شود که اگر تعداد فردی از ورودی ها تغییر کند، سبب تغییر خروجی می شود. لذا نیاز است تعداد حالات فرد ورودی را اشتراک و در ادامه با سایر ورودی های صفر که اجتماع گرفته ایم ، از هم کم کنیم. البته همانند حالت قبل ، باید توجه داشت که در نهایت لیست اشکال خود سیم خروجی را نیز به لیست اشکال های آن سیم اضافه کنیم.
\end{enumerate} 

در ادامه کد مربوط به 
\lr{\textbf{Deductive Fault Simulation}}
، آورده شده است:

\small {\lr{\lstinputlisting[numbers=left, breaklines=true, language=Python, firstline=12, lastline=489]{../project/operations/fault_simulation_operations/fault_simulation_deductive_operation.py}}}

