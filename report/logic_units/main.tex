\section{واحد های منطقی و محاسبات مربوطه بر روی آن ها}

\subsection{گیت ها و سیم ها}
هر مدار منطقی موجود اعم از فایل های 
\lr{.bench}
که در پروژه موجود است، حاوی دو عنصر کلیدی هستند:
\begin{inparaenum}
	\item 
	\textbf{گیت های منطقی}
: در واقع گیت های منطقی جزء جدایی ناپذیر یک مدار منطقی به شمار می آیند. در این پروژه هر گیت منطقی می تواند یکی از حالات 
	\lr{AND, NAND, OR, NOR, XOR, XNOR, FANOUT}
	باشد. 
	\item
	\textbf{سیم ها}
	: که به عنوان پلی بین گیت های منطقی به حساب می آیند.
\end{inparaenum}
، می باشد.

از آنجایی که در فایل های 
\lr{bench}
فقط در مورد گیت ها و ورودی های آن صحبت کرده، لذا نیاز است پس از اینکه گیت های منطقی را از ورودی خواندیم، در ادامه گیت های 
\lr{FANOUT}
را تشکیل دهیم. پس از چنین روندی نیاز است از یک ساختار داده 
\lr{WIRE}
جهت اتصال بین گیت ها استفاده کنیم. اگر بخواهیم خیلی جزئی به بررسی اینکه چگونه گیت های 
\lr{FANOUT}
را تشکیل می دهیم، صحبت کنیم، این روند به این صورت است که اگر یک گیت دارای بیش از چند خروجی داشته باشد، آن گیت را به یک گیت
\lr{FANOUT}
متصل می کنیم و خروجی های آن گیت را به عنوان خروجی های 
\lr{FANOUT}
در نظر می گیریم.

کدی که برای هر گیت موجود است، به شکل زیر است. البته توجه داشته باشید که این روند برای هر گیت تقریبا یکسان است و تنها در تابع 
\lr{\_specific\_validation}
متفاوت هستند و این تابع مسئولیت این را دارد که یک اعتبار سنجی برای تعداد ورودی و خروجی های گیت داشته باشد:
\small {\lr{\lstinputlisting[numbers=left, breaklines=true, language=Python, firstline=18,lastline=152]{../project/logic_units/gates/gate.py}}}

در واقع روند تولید هر گیت به این صورت است که در ابتدا مقدار هر گیت به صورت نامشخص یا 
\lr{UNKNOWN}
تعریف می شود. در ادامه پس از محاسبات مربوطه گیت بر اساس سیم های ورودی اش، خروجی گیت محاسبه و به سیم های خروجی اش منتشر می شود. 

در ادامه به بررسی ساختار هر 
\lr{WIRE}
می پردازیم:
\small {\lr{\lstinputlisting[numbers=left, breaklines=true, language=Python, firstline=6, lastline=73]{../project/logic_units/wires/wire.py}}}

همانطور که گفته شد، پس از اینکه گیت ها تشکیل شدند، نیاز است از طریق سیم ها یا همان
\lr{WIRE}
ها اتصال بین آن ها برقرار شود و این امر مشابه با گیت ها، ابتدا مقادیر سیم ها نیز به صورت نامشخص یا 
\lr{UNKNOWN}
تعریف می شود و پس از اینکه مقادیر گیت محاسبه شد، خروجی آن گیت بر روی این سیم ها منتشر می شود و این سیم ها مقادیر گیت های سطح بعدی را تامین می کنند.

\subsection{محاسبات منحصر به فرد هر گیت}
حال در ادامه به بررسی محاسبات برای هر گیت می پردازیم. کد محاسبات مربوط به هر گیت به شکل زیر است: 

\small {\lr{\lstinputlisting[numbers=left, breaklines=true, language=Python, firstline=10, lastline=194]{../project/operations/logic_operation.py}}}

روند کلی این کد به صورت بازگشتی تعریف شده است. بدین معنا برای مثال گیت 
\lr{AND}
دارای 
\lr{n}
ورودی باشد، خروجی آن گیت به صورت روبرو بدست می آید:
$AND(a[n]) = AND(a[0], AND(A[1:n]))$


 اگر فرض کنیم گیت های اصلی به صورت گیت های 
\lr{AND}
و 
\lr{OR}
و
\lr{NOT}
هستند، گیت NAND به صورت NOT کردن AND ورودی ها بدست می آید و به صورت مشابه گیت 
NOR
به صورت NOT کردن OR ورودی ها بدست می آید. از طرفی گیت XOR دو ورودی (حاوی ورودی های a و b) به صورت 
$XOR(a,b) = \overline{a} \times b + a \times \overline{b}$
بدست می آید. همچنین گیت 
\lr{XNOR}
به صورت NOT کردن XOR ورودی ها بدست می آید.

از این رو در ادامه به بررسی جدول درستی گیت های 
\lr{AND, OR , NOT}
به ازای مقادیر 
\lr{0,1,Z,U}
می پردازیم. 

جدول درستی گیت
\lr{NOT}
به صورت زیر تعریف می شود:

\begin{latin}
\begin{center}
	\begin{tabular}{ |p{3cm}|p{3cm}| }
		\hline
		\multicolumn{2}{|c|}{NOT Operation} \\
		\hline
		\textbf{a} & \textbf{NOT(a)} \\
		\hline
		0 & 1 \\
		1 & 0 \\
		Z & Z \\
		U & U \\
		\hline
	\end{tabular}
\end{center}
\end{latin}

همچنین جدول درستی گیت دو ورودی AND به صورت زیر تعریف می شود: 

\begin{latin}
\begin{center}
\begin{tabular}{ |p{3cm}|p{3cm}|p{3cm}| }
\hline
\multicolumn{3}{|c|}{AND Operation} \\
\hline
\textbf{a} & \textbf{b} & \textbf{AND(a,b)} \\
\hline
0 & 0 & 0 \\
0 & 1 & 0 \\
0 & Z & 0 \\
0 & U & 0 \\
1 & 0 & 0 \\
1 & 1 & 1 \\
1 & Z & Z \\
1 & U & U \\
Z & 0 & 0 \\
Z & 1 & Z \\
Z & Z & Z \\
Z & U & U \\
U & 0 & 0 \\
U & 1 & U \\
U & Z & U \\
U & U & U \\
\hline
\end{tabular}
\end{center}
\end{latin}

همچنین جدول درستی گیت 
\lr{OR}
دو ورودی به شکل زیر است:

\begin{latin}
\begin{center}
\begin{tabular}{ |p{3cm}|p{3cm}|p{3cm}| }
\hline
\multicolumn{3}{|c|}{OR Operation} \\
\hline
\textbf{a} & \textbf{b} & \textbf{OR(a,b)} \\
\hline
0 & 0 & 0 \\
0 & 1 & 1 \\
0 & Z & Z \\
0 & U & U \\
1 & 0 & 1 \\
1 & 1 & 1 \\
1 & Z & 1 \\
1 & U & 1 \\
Z & 0 & Z \\
Z & 1 & 1 \\
Z & Z & Z \\
Z & U & U \\
U & 0 & U \\
U & 1 & 1 \\
U & Z & U \\
U & U & U \\
\hline
\end{tabular}
\end{center}
\end{latin}

همانطور که اشاره شد، سایر گیت ها را می توان از طریق این سه گیت اصلی
\lr{NOT‌, AND , OR}
بدست آورد و همچنین برای گیت های بیش از دو ورودی می توان از مکانیزمی بازگشتی گفته شده، استفاده کرد.

در ادامه نیاز است ورودی ها را از فایل خوانده و به شبکه گیت ها تزریق کنیم و در هر گام خروجی گیت های آن سطح را محاسبه کنیم تا در نهایت به خروجی برسیم. به پراکندگی کد این قسمت، از آوردن کد آن صرف نظر کرده و تنها به همین توضیحات مختصر بسنده می کنیم.


