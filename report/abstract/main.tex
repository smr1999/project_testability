\section{مقدمه و شرح مسئله}

در این گزارش قصد داریم به بررسی فاز دوم پروژه درس آزمون‌پذیری بپردازیم. در این پروژه فایل‌هایی با پسوند 
\lr{bench}
که حاوی اطلاعاتی در مورد ورودی، خروجی و همچنین گیت‌های مدار است را خوانده و موارد زیر را اجرا می‌کنیم:

\begin{enumerate}
	\item 
	در گام اول ابتدا بر اساس الگوریتم استنتاجی شبیه‌سازی اشکال 
	\LTRfootnote{Deductive fault simulation}
	که در فاز قبلی به طور مفصل به آن پرداخته شد،
	به تعداد 
	$2^n$
	، که n همان ورودی‌های مدار است (روش جامع
	\LTRfootnote{Exhaustive}
	) ، ورودی به مدار تزریق می‌کنیم و در هر گام اشکالات چسبیدگی تکی
	\LTRfootnote{Stuck-at-fault}
	ای که در خروجی مدار پدیدار شده‌اند را ذخیره کرده و در نهایت این بردارهای تست را به همراه اشکالات شناسایی شده توسط هر بردار تست را در یک فایل با پسوند 
	\lr{csv}
	ذخیره می‌کنیم تا در نهایت بتوانیم به 
	\lr{Fault Dictionary}
	دست یابیم.

	\item 
	در گام بعدی بر اساس قوانین موجود در معادل‌بودن اشکال 
	\LTRfootnote{Fault Equivalence}
	تعداد اشکالات موجود در مدار را کاهش داده و بر اساس این قوانین مجدداً 
	\lr{Fault Dictionary}
	را تشکیل می‌دهیم. همچنین در این گام بردار‌های تست ضروری یا 
	\lr{Essential Test Vectors}
	را نیز مشخص کرده و همه را در یک فایل
	\lr{csv}
	دیگر ذخیره می کنیم.
\end{enumerate}
