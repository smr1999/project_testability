\section{مقدمه و شرح مسئله}

در این گزارش قصد داریم به بررسی پروژه اول درس آزمون پذیری بپردازیم. در این پروژه فایل هایی با پسوند 
\lr{bench}
که حاوی اطلاعاتی در مورد ورودی، خروجی و همچنین گیت های مدار است را خوانده و در ادامه دو گام اصلی را انجام می دهیم: 
\begin{inparaenum}
	\item 
	در گام اول یک ورودی که می تواند مقادیر 0، 1 ، U و Z هستند را به مدار اعمال می کنیم و مقادیر نت ها و خروجی ها را مشخص می کنیم.
	\item 
	در گام بعدی یک ورودی که می توانند مقادیر باینری 0 و 1 هستند را به مدار تزریق می کنیم و اشکالات مدار که از نوع 
	\lr{stack-at}
	 هستند را بر اساس الگوریتم استنتاجی 
	 \LTRfootnote{Deductive algorithm}
	و قواعد مجموعه ها در هر نت و همچنین خروجی های مدار، مشخص می کنیم.
\end{inparaenum}
